\begin{abstract}
We propose a recursive quantum cosmological model in which the universe evolves through entangled cycles, preserving memory via coherence propagated across spacetime bridges. The framework unifies three principles: (1) the discrete geometry of loop quantum cosmology, (2) Einstein-Rosen bridge thermodynamics as conduits for memory transfer, and (3) non-Markovian decoherence regulated by memory-sensitive kernels.

We define a recursive configuration state vector \( \phi \in \mathbb{R}^{12} \), spanning spacetime dimensions (L, W, H, T), light-field indices (R, G, B), signal amplitude, quantum interference, a recursive index, observer entanglement, and a universe-specific identifier. Transitions between cycles are governed by a generalized XOR-like operator \( \phi_{n+1} = \phi_{U_1} \oplus \phi_{U_2} \), which encodes observer-modulated geometric interference.

A recursive action functional \( \mathcal{A}_n[\phi] \) is introduced, combining classical dynamics, decoherence constraints, and entropic penalties. Observable predictions include low-\( \ell \) CMB power suppression, gravitational wave spectral echoes, and entropy conservation across cosmological bounces. This framework offers a memory-driven mechanism for coherence retention in a self-observing universe.
\end{abstract}
