\section*{11. Recursive Variational Principles and Symmetry of Action}

\subsection*{11.1 Recursive Variational Structure}

Traditional physics derives motion from an action principle~\cite{feynman1965feynman}. But the universe does more than move—it remembers. We propose the \textbf{Recursive Action Principle (RAP)}:

\[
\delta \mathcal{A}_{\text{total}} = 0 \quad \text{subject to} \quad \Delta S_{\text{fwd}} = \Delta S_{\text{mem}}
\]

This extends traditional variational formulations by coupling dynamical evolution to informational memory~\cite{lloyd_quantum_1988,zurek2003decoherence}. It does not replace field-theoretic principles within each cycle—it layers coherence constraints across cycles.

\subsection*{11.2 Cycle-Level Action Terms}

Each cycle \( n \) is governed by an action:

\[
\mathcal{A}_n = \int dt \left[ \frac{1}{2} G^{IJ} \dot{q}_I \dot{q}_J - V(q) + D[\rho_n] - M_n(q_n, q_{n-1}) \right]
\]

where:
\begin{itemize}
    \item \( G^{IJ} \): Field-space metric from LQG minisuperspace~\cite{ashtekar2006quantum}.
    \item \( D[\rho_n] = -\text{tr}(\rho_n \ln \rho_n) \): Decoherence entropy~\cite{breuer2002theory}.
    \item \( M_n = \lambda_E e^{-\beta \Delta S_n} |\langle \Psi_n | \Psi_{n-1} \rangle|^2 \): Memory coupling~\cite{grigolini1999coherence}.
\end{itemize}

This captures the dynamical and informational evolution within a cycle.

\subsection*{11.3 Emergent Time and Observer Boundary}

Time arises from memory flow. Observation acts as a boundary constraint~\cite{zurek2009quantum}:
\[
\delta \Psi_n \big|_{\Sigma_{\text{obs}}} = O_n \Psi_n
\]

The operator \( O_n \) varies with context—it projects onto the observable subspace defined by entanglement structure, not necessarily by consciousness~\cite{tegmark_consciousness_2015}. It reflects when and how memory collapses into classicality.

Entropy production and memory retention are defined as:

\[
\Delta S_{\text{fwd}} = S[\rho_n] - S[\rho_{n-1}], \quad \Delta S_{\text{mem}} = -\lambda_S \ln |\langle \Psi_n | \Psi_{n-1} \rangle|
\]

\subsection*{11.4 Recursive Duality: Particle and Wave}

Reality cannot be described by particle evolution or wave propagation alone. It requires both:

\begin{itemize}
    \item \textbf{Lagrangian} \( \mathcal{L}(q_n, \dot{q}_n) \): Local particle dynamics~\cite{feynman1965feynman}
    \item \textbf{Action} \( \mathcal{A}_n \): Global wave interference and coherence~\cite{hartle1983wave}
\end{itemize}

This duality mirrors the wave-particle duality of quantum mechanics—resolved through recursive evolution.

\subsection*{11.5 Unified Principle and Constraint Formulation}

The recursive action is constrained by coherence balance:

\[
\delta \mathcal{A}_{\text{total}} + \lambda_C \delta(\Delta S_{\text{fwd}} - \Delta S_{\text{mem}}) = 0
\]

The Lagrange multiplier \( \lambda_C \) governs the relative strength of memory conservation. Analogous to Newton’s gravitational constant \( G \), it encodes the “gravitational pull” of coherence across time~\cite{gellmann1994complex}.

This principle defines the recursion of universes. Future work will derive Euler-Lagrange equations in recursive field space.

\textit{The action is not only over paths in spacetime. It is over the very act of remembering.}


