\section*{Appendix F: Lithium as a Marker of Recursive Stability}

\subsection*{F.1 Cosmological Lithium Anomaly}

In standard big bang nucleosynthesis (BBN), the primordial abundances of light elements—hydrogen, helium, and lithium—are well-predicted by the baryon-to-photon ratio inferred from CMB observations. While hydrogen and helium match observations closely, a persistent discrepancy exists in the abundance of lithium-7: theoretical predictions exceed observational data by a factor of 2–3~\cite{fields2020bbn}. This unresolved tension, known as the \textit{cosmological lithium problem}, challenges the completeness of standard cosmological nucleosynthesis models.

\subsection*{F.2 Recursive Interpretation: A Residue of Prior Decoherence}

In the context of a recursive cosmological framework, the lithium anomaly may be reinterpreted not as a flaw in standard nuclear modeling, but as a subtle trace of coherence filtering across cosmological cycles. Given that lithium forms at a delicate energetic threshold between bound-state fusion and decay, its abundance may encode the thermodynamic memory of prior cycle entanglement—preserved imperfectly through the bounce.

We hypothesize that lithium acts as a \textit{material witness} to decoherence: a physical element whose observed abundance reflects the entropy filtering imposed by the decoherence kernel \( D(\tau, E) \). In this view, the discrepancy in lithium-7 may not signal an error, but an imprint of recursive entropy regulation—consistent with our model’s claim that memory and structure persist probabilistically across bounces, weighted by coherence fitness.

\subsection*{F.3 Neurobiological Analogy: Lithium in Mental Stability}

Intriguingly, lithium also plays a unique role in neurobiology. As a pharmacological agent, lithium is the most effective known mood stabilizer in the treatment of bipolar disorder—a condition characterized by oscillatory instability in emotional and cognitive coherence~\cite{malhi2017lithium}. Lithium reduces the amplitude and frequency of affective swings, enhances synaptic plasticity, and modulates neurochemical pathways associated with memory and identity.

Although purely analogical, the parallel is striking: in both the brain and the cosmos, lithium appears as a \textit{regulator of recursive oscillation}. In psychiatric treatment, it stabilizes cycles of coherence and collapse. In our cosmological model, it may play a related role—marking where coherence was marginally retained or where recursive entropy failed to fully erase prior structure.

\subsection*{F.4 Interpretive Scope}

We emphasize that this appendix is interpretive and speculative. No predictive claim is made regarding the absolute abundance or quantum state of lithium. However, the convergence of two anomalous domains—cosmological lithium deviation and lithium's neurostabilizing effect—invites further exploration of possible structural or symbolic resonances. If the self-remembering universe preserves patterns across scales, lithium may be among the few elements that encode that memory both materially and metaphorically.

