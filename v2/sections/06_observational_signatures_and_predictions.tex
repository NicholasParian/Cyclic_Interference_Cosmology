\section*{6. Observational Signatures and Predictions}

The recursive quantum cosmology model proposed herein leads to several distinct observational signatures, arising from its interference-based structure, memory-preserving kernels, and entropy-regulated decoherence dynamics. These signatures manifest across cosmic microwave background (CMB) anisotropies, the stochastic gravitational wave (GW) background, and entropy dynamics near bounce transitions.
\subsection*{6.1 Collapse and Coherence Darwinism}

The universe, like memory, is fragile. It does not always remember. Coherence is not guaranteed. When the fidelity of recursion fades, when the entanglement between cycles weakens, the wavefunction collapses --- not locally, but cosmologically~\cite{zurek_decoherence_2003,tegmark_consciousness_2000}.

Decoherence is not destruction. It is transition. The density matrix \( \rho_n \), once pure, becomes mixed. Interference becomes noise. The system forgets its own path. We quantify this loss through entropy~\cite{schlosshauer_decoherence_2007}:

\[
D[\rho_n] = -\mathrm{tr}(\rho_n \ln \rho_n)
\]

The entropy does not vanish --- it shifts. What one cycle loses, another may inherit. But only if the fidelity survives~\cite{lloyd_quantum_1988}.

To formalize this, we define a \textbf{coherence fitness functional}:

\[
F_n = \alpha_C \cdot \mathrm{tr}(\rho_n^2) - \alpha_S S_n + \alpha_M M_n
\]

This functional governs recursive propagation. When \( F_n \geq F_{\text{crit}} \), coherence continues --- the system remembers. When \( F_n < F_{\text{crit}} \), memory collapses and reinitialization occurs.

Not all collapse is total. Some cycles retain partial structure --- memory fragments, entropic residues, incomplete inheritance. Collapse is not a cliff, but a curve~\cite{zurek_quantum_2009}.

To model this, we define the \textbf{Survival Probability of Recursion}:

\[
P_{\text{survive}}(F_n) = \frac{1}{1 + e^{\kappa (F_{\text{crit}} - F_n)}}
\]

This smooth logistic function captures the probabilistic nature of recursion retention. Systems with near-threshold coherence may propagate partially, with diminished memory or altered structure.

We define the target of recursive convergence as the \textbf{Convergence Field} \( \mathcal{C}^*(\phi) \): an emergent attractor shaped by recursive fidelity, coherence, and entropy balance. It is not static, but reinforced by the memory of the system itself.

Each cycle contains variation. Some are closer to the attractor, others drift. The ones that retain coherence --- that resonate with the prior state --- persist. Others decay. Not because they were less ``fit,'' but because they were less coherent.

This principle --- \textbf{Recursive Coherence Darwinism} --- describes the selection pressure acting on entire universes. It is not survival of the fittest, but survival of the most resonant~\cite{darwinism_cosmology_2018,tegmark_mathematical_2008}.

\vspace{1em}
\noindent
\textit{“The universe doesn’t allow perfection.”}~\cite{hawking_brief_1988} \\
\hfill --- Stephen Hawking

\vspace{1em}
\noindent
\textit{“The act of forgetting is not the end of memory, but its transformation…”}~\cite{greene_fabric_2004}
 \\
\hfill ---  Brian Greene

\vspace{2em}
% Placeholder: Insert visual of F_n over n, bifurcation to convergence field
% Placeholder: Insert visual metaphor of recursive attractor basin (Convergence Field)

In human-scale analogs, the memory term \( M_n \) may encode not only quantum overlap but also emotional coherence --- the capacity of the system to reflect, respond, and retain. In this sense, coherence is both mathematical and meaningful~\cite{penrose_emperors_1994,tononi_consciousness_2004}.

Recursion does not end in collapse. It ends in re-formation. The loss of coherence does not erase the universe --- it seeds its next becoming.

\noindent\textit{Note:} For a complementary geometric interpretation of relativistic coherence collapse, see Appendix D.


\subsection*{6.2 CMB Power Suppression at Large Angular Scales}

Quantum interference with prior-cycle memory states, encoded in the initial condition kernel \( K(\phi, \phi') \), introduces a phase coherence constraint on long-wavelength cosmological perturbations. This generically leads to suppression of power at low multipole moments (\( \ell < 30 \)). The observed suppression in the Planck satellite data is consistent with this framework.

The transition kernel contributes an effective Gaussian cutoff to curvature perturbations:
\[
P(k) \rightarrow P(k) \cdot \exp\left[-\frac{(\Delta \phi - \phi_*)^2}{2\sigma_\phi^2}\right]
\]
where \( \phi_* \) is the most probable overlap configuration from prior cycles. This modifies the Sachs-Wolfe plateau in the angular power spectrum \( C_\ell \).

\subsection*{6.3 Gravitational Wave Spectrum Modulation}

Compactified geometry across bounces induces modulated expansion rates, generating characteristic features in the stochastic GW background. If the bounce dynamics are quasi-periodic due to Calabi-Yau moduli stabilization, then resonance-like oscillations emerge in \( \Omega_{\text{GW}}(f) \).

Expected frequencies:
\[
f_j \sim \frac{j}{L_c}, \quad j \in \mathbb{Z}^+
\]
where \( L_c \) is the compactification scale. These modulations appear as dips or harmonics superimposed on a scale-invariant background, potentially observable by LISA or future interferometers.

\subsection*{6.4 Entropy Conservation Across Cosmological Bounces}

The non-Markovian memory kernel \( D(\tau) \), when integrated over a full cycle, enforces a bounded entropy flux. The recursive framework ensures that the entanglement entropy \( S(t) = -\mathrm{tr}(\rho \log \rho) \) asymptotically saturates, reflecting a steady-state attractor across successive bounces.

This implies a quasi-conservation law:
\[
S_{n+1} \lesssim S_n + \delta S_{\text{Hawking}} - \delta S_{\text{decohere}}
\]
where memory retention and Hawking-like radiation contribute opposing entropy flows. Observable consequence: retention of low-entropy structure in initial conditions after bounce, testable via anomalies in early-universe correlations.

\subsection*{6.5 Summary}

These predictions distinguish the proposed model from standard inflationary cosmology. In particular:
\begin{itemize}
    \item CMB suppression at low \( \ell \) arises naturally without fine-tuned inflationary initial conditions.
    \item GW spectral modulation provides a signature of cyclic geometry and compactification.
    \item Entropy quasi-conservation implies memory persistence, providing a falsifiable constraint on inter-bounce dynamics.
\end{itemize}

Future observational campaigns—particularly in low-\( \ell \) polarization (e.g., LiteBIRD) and stochastic GW detection—can test these signatures.

