\section*{5. Recursive Quantum Action and State Interference}
\label{sec:recursive-action}

We define the recursive evolution of the universe through a configuration state vector \( \phi \), encoding the geometric, quantum, and observational parameters of a cosmological cycle. The transition kernel \( K(\phi, \phi') \) governs the amplitude between successive cycle states, constrained by coherence memory, entropy penalties, and observer entanglement.

\subsection*{5.1 State Vector Definition}

The cosmological configuration state is defined as:
\begin{equation}
\phi = \{L, W, H, T, R, G, B, A, Q, I, U, O\}
\end{equation}

\begin{itemize}
  \item \( L, W, H, T \): Classical spacetime dimensions.
  \item \( R, G, B \): Example encoding of field values via light-based indices; generalized to all quantum degrees of freedom.
  \item \( A \): Curvature-coupled scalar field amplitude.
  \item \( Q \): Quantum interference parameter (phase alignment, coherence strength).
  \item \( I \): Recursive index (cycle count).
  \item \( U \): Universe path label (within ensemble).
  \item \( O \): Observer entanglement tensor, modeled as an internal memory field entangled with \( \phi \), updated each cycle via learning-like adaptation.
\end{itemize}

\subsection*{5.2 Interference Operator \( \oplus \)}

We define a recursive interference operator \( \oplus \), representing a blended XOR of state information modulated by phase interference:
\begin{equation}
\phi_{n+1} = \phi_{U_1} \oplus \phi_{U_2} = \frac{1}{Z} \cdot \text{XOR}_q(\phi_{U_1}, \phi_{U_2}) \cdot \cos(\Delta \theta)
\end{equation}

\begin{itemize}
  \item \( \text{XOR}_q \): Quantum-blended XOR across all fields (logical for discrete, interpolated for continuous variables). 
  \item \( \cos(\Delta \theta) \): Phase interference term where \( \Delta \theta \) is derived from curvature phase shift and entanglement geometry between \( \phi_{U_1} \) and \( \phi_{U_2} \).
  \item \( Z \): Coherence normalization factor depending on memory fidelity and decoherence scale.
\end{itemize}

\subsection*{5.3 Recursive Action \( \mathcal{A}_n \)}

The recursive total action is:
\begin{equation}
\mathcal{A}_n = \sum_{k=1}^{n} S[\phi_k] + \lambda_E \cdot D(\tau_k, E) - \gamma \cdot S_{\text{ent}}(\phi_k)
\end{equation}

\noindent
Where:
\begin{itemize}
  \item \( S[\phi_k] \): Einstein-Hilbert (or LQG-modified) action for configuration \( \phi_k \).
  \item \( D(\tau_k, E) \): Memory kernel with delay \( \tau_k \) and entanglement eigenvalue \( E \); governs memory retention across cycles.
  \item \( S_{\text{ent}}(\phi_k) \): Entropy penalty term reflecting decoherence; bounded by
  \[
    S_{\text{ent}} < S_{\text{max}}(E, n) = \frac{1}{E^2 + n^2}
  \]
  based on the system's coherence capacity and cyclic duration.
\end{itemize}

\subsection*{5.4 Observable Predictions}

Observable effects emerging from this recursive formalism include:
\begin{itemize}
  \item \textbf{Low-\( \ell \)} CMB suppression:
  \[
    C_\ell^{\text{rec}} \propto \cos^2(\Delta \theta) \cdot e^{-I/\sigma_I^2}
  \]
  due to destructive memory interference and cycle count damping.

  \item \textbf{Gravitational wave echoes}:
  Logarithmically spaced echo signatures emerge from boundary-induced memory delays in \( K(\phi, \phi') \); each "bounce" reflects partial coherence, producing a delayed GW mode at:
  \[
    f_j \sim \frac{1}{\tau_M} \cdot e^{-j \Delta}
  \]

  \item \textbf{Entropy bounds from coherence decay}:
  Decoherence growth is suppressed by entanglement structure:
  \[
    S_{\text{ent}} \ll \frac{A(\phi)}{4G} \quad \text{only if } E > E_{\text{crit}}
  \]
  ensuring coherence-dominated evolution.
\end{itemize}

See Appendix~\ref{appendix:D} for detailed structure of \( \text{XOR}_q \), derivation of \( \Delta \theta \), and memory propagation constraints tied to observer entanglement \( O \).
