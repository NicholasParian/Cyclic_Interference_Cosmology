\section*{12. Interpretation, Limitations, and Future Directions}

\subsection*{12.1 Interpretation of the Framework}

This model reinterprets cosmology through the lens of recursion. It proposes that the universe is not merely a system that expands---it is a system that remembers. Memory is encoded in quantum overlaps across cycles. Coherence is not a byproduct; it is the driving principle. Entropy flows forward, but memory anchors the past~\cite{zurek2003decoherence}.

Time, in this view, is recursive. Each cycle is both an effect and a witness. The wavefunction evolves not only with dynamics, but with interference. The attractor is not a solution---it is a learned resonance, filtered through entropy and sustained by coherence~\cite{gellmann1994complex, hartle1983wave}.

This is not a metaphysical claim. It is a physical structure. A constrained action across cycles. A symmetry not of space or time, but of recurrence.

\subsection*{12.2 Model Limitations}

This framework is a proof of principle, not a completed theory of quantum gravity. Several elements are formulated at the effective level:

\begin{itemize}
  \item The recursive Lagrangian is constructed using field-theoretic approximations rather than derived from full quantum gravity~\cite{rovelli2004quantum}.
  \item The Einstein-Rosen bridge action is posited with holographic and entropic couplings but not derived from a complete path-integral formulation~\cite{maldacena2013cool, vanraamsdonk2010entanglementgeometry}.
  \item The 12D Hilbert space, while mathematically consistent, remains speculative. It is not required for the model to function, but is proposed to describe boundary conditions during coherence collapse~\cite{tegmark2008mathematical}.
\end{itemize}

No definitive claim is made about first-cycle conditions, the origin of time, or the completeness of this system. These remain open.

\subsection*{12.3 Comparison to Other Cosmological Models}

This framework complements and extends ideas found in several other paradigms:

\begin{itemize}
  \item \textbf{Conformal Cyclic Cosmology (CCC)} proposes recursive expansion through conformal mapping~\cite{penrose2010cycles}. Like CCC, this model assumes no final entropy state. But unlike CCC, it provides a mechanism for quantum memory and coherence via ERB-mediated coupling~\cite{maldacena2013cool}.
  \item \textbf{Loop Quantum Cosmology (LQC)} provides bounce dynamics and discreteness at the Planck scale~\cite{ashtekar2006quantum}. This framework builds on LQC but introduces memory transfer, entropy constraints, and coherence kernels that allow cycles to influence each other.
  \item \textbf{Inflationary Cosmology} explains large-scale structure and flatness but treats initial conditions as externally given~\cite{guth1981inflationary}. This model internalizes those conditions through a recursive variational principle~\cite{lloyd_quantum_1988}.
\end{itemize}

The differences lie not in the equations of motion alone, but in the structure of what is preserved.

\subsection*{12.4 Scope of the Observer}

This paper does not attempt to define the observer. The operator \( O_n \), which imposes boundary conditions during decoherence, is treated formally. It may represent an entangled subsystem, a projection event, or something emergent~\cite{zurek2009quantum}.

Further exploration into the role of observation, intention, and decoherence-induced classicality is anticipated~\cite{tegmark_consciousness_2015}. However, to preserve focus, this paper defers that inquiry to future work.

\subsection*{12.5 Future Work and Research Pathways}

Key areas of investigation include:

\begin{itemize}
  \item Deriving full Euler-Lagrange equations from the recursive Lagrangian formalism.
  \item Simulating the attractor and memory feedback numerically.
  \item Refining the decoherence kernel \( D(\tau, E) \) and understanding its collapse conditions~\cite{grigolini1999coherence}.
  \item Clarifying the nature of the entanglement eigenvalue \( \lambda_n \) and its physical observability.
  \item Establishing curvature thresholds \( R_{\text{crit}} \) for coherence failure and recursive reset~\cite{ashtekar2011loop}.
  \item Exploring the structure and logical basis of the 12D Hilbert space as a meta-informational boundary~\cite{tegmark2008mathematical}.
\end{itemize}

This framework remains open, testable, and incomplete by design. It aims to provide structure, not certainty. And where it leaves questions open, it leaves pathways for the next step.

